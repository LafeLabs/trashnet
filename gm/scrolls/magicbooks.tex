

The media which drives this network is the Magic Book. The Magic Book is
a format for electronic books which lends itself to easy replication
across the Internet and easy editing. We use the term ``magic'' as in
all other places in this work to mean that people replicate them
themselves. That is, with simply copy/pasting links and clicking on them
in browsers anyone on a network can copy any book from any server to any
other server.

This is not like Google Docs where documents are attached to ``users''
who log into a cloud server controlled by Google. We use cloud hosting
for public-facing web pages, but they are all able to be read by anyone
anywhere in the world freely without any log ins or passwords. And all
these books are created, edited, and shared on the local Pibrary
networks hosted on physical Raspberry Pi based web servers which are
shared freely in our local physical community.

Also, we are using ``book'' as a metaphor. What is a book exactly? How
is it different from other media? A book can be physical or digital, can
be a private document or public. But what distinguishes it from things
like articles or news is that it is self-contained and encapsulates a
large body of knowledge in a coherent whole. A book can evolve over time
and can get re-written but it has a fundamentally different structure
from the news feeds which dominate social media today. Also, we
distinguish these books from wiki's like Wikipedia. Wikis are databases
of articles. These articles are the fundamental element of the whole
thing, and are not organized into book structures. It is hard to write
down a clear definition which distinguishes an article from a book, but
for our purposes a book is a collection of chapters, each of which is a
text document, and all of which add up to some coherent whole.

All books are released by their authors into the Public Domain with no
restrictions whatsoever. We do not use the kind of restricted licenses
favored by Creative Commons or the Free Software Foundation, but
explicitly release books under the Public Domain.

Books, like everything else in the work described here, are
self-replicating sets. That is, collections of things which all
replicate as a set easily by anyone on the network. The main element of
these sets are called scrolls and these are just text documents in the
Markdown format. Markdown is a very simple text format which is used in
a wide range of online content, which in its simplest form is just raw
text, but has a few simple additions like using asterisks for italic,
double asterisks for bold, and number signs for headings. While using a
markup language like this with a little bit of syntax is in some ways
more complex than the completely point-and-click driven editors like
Word or Google Docs, this is designed to make the documents compatible
with pure-text formats, which is important for making them easy to
replicate and edit as we move them freely across our network. We believe
that the usability cost of Markdown is worth it for the usability gain
of being plain human readable text. The Scroll format used in this work
is Markdown converted on-the-fly into HTML using the open source
JavaScript library showdown.js. This allows us to set format parameters
like font, size, color, and how text fits in a screen using standard web
development methods, adapting the same text document to any look and
feel or screen size we want(a huge advantage over pdf). This reliance on
standard web development methods allows us to have our format work well
on all web enabled devices from mobile to tablet to desktop to big
screens and of course our free public Raspberry Pi computers without any
software other than the browser.

While our primary media format is in the web browser, it is also useful
to be able to generate physical bound books, also for free distribution.
There are a number of ways to do this, but the one we recommend and are
using for this work is LaTeX(pronounced LAY-TECH or LAH-TECH, the ``X''
is meant to represent the Greek letter ``Chi''), and document formatting
system developed for the typesetting of technical work. Like Markdown,
this is a human-readable text format in which standard text characters
are used to indicate to the computer how format will work. For example,
while Markdown uses asterisks around a word for italic, LaTeX uses a
backslash command ``\emph" along with curly brackets around whatever
goes in italic. The most important thing about using LaTeX is that for
when we create more technical works diving into the physics, engineering
and math needed to build the world of full Trash Magic that it makes
that easy. This system is already widely used by technology creators and
scientists so while it has a steep learning curve it is useful for the
experts we are inviting into this system to create technical books.
Also, it is compatible with a number of other web-based systems of
technical documentation like the Jupyter Notebooks which are an almost
universal means of communication now in applied sciences where
calculations are done on data using Python or other popular data science
languages like R. Another widely used and open source Javascript
library(Mathjax.js) allows us to optionally turn on this math
typesetting in the Markdown-driven scroll documents as well, so
technical books can be written entirely in the Pibrary format and then
moved to LaTeX with the math formatting staying the same. Conversion
from Markdown to LaTeX can be enabled with Pandoc, the''swiss army
knife" of document formats(see pandoc.org for details). Once a book is
in the LaTeX format it is compiled into .pdf in whatever book size is
appropriate. We generally use this to compile to two formats: the letter
size in the US or A4 in metric countries for printing on standard home
or office printers to bind in three ring binders and the 6x9 inch format
for binding from print-on-demand publishers. We use Lulu Press(lulu.com)
to create the bound copies in various formats.

The exact means by which books are replicated will be discussed
elsewhere, but essentially it is all based on building links to scripts
which can be run from a browser which fetch lists of files and use that
list to fetch all the files. The best way to learn how to do this is by
example, and the replication of this work will involve directly showing
people how to do this in person, via video, and via real time online
communication we will be setting up in the coming months.

So far we have discussed the format of the Magic Books but not the
purpose or what books we will share first. The purpose of the library of
books we are creating here is to be a repository of all the knowledge
needed to build full Trash Magic. This means we need to create a culture
with everything that goes along with that: history, philosophy,
politics, technology, science, math, and all the wisdom required to be
stewards of the land we are a part of. This system of books also needs
to self-support. This means that as a social media platform it needs to
generate economic value measured initially in money which can provide
material support to those of us creating and replicating the network.

The beginning of this library is the books created by the author of this
book, Trash Robot. This includes the Trash Magic Manifesto, the Trash
Magic Action Coloring Book, and the first Book of Geometron, as well as
this book, Geometron Magic. Trash Robot is also in the process of
creating another book, Trash Physics, which is part memoir, part
criticism of the structure of modern physics, and partly the start of a
whole library of physics texts based on the principles here.

The way the library of Magic Books we describe here will become
self-sustaining is by documenting the commerce in local communities in a
deep and organized way that no existing resource does. This is not just
a business directory. It is the creation of a new level of social
networking in physically local spaces that does not exist on today's
Internet. We will work with local people to create books on local
history, local culture, the local economy, the local government, local
mutual aid and outreach organizations, local libraries, local religious
institutions, and compile all of into books which are shared on our
system. Again, this is not a wiki covering existing things. This is a
library, creating new deeper connections than exist today, diving deeper
into history and culture than the existing Web does. It is also not
news. We aim to create new social links in physically local spaces with
our system which enable people to engage in new commerce with each other
locally. We call these books the Books of the Street, where the
``street'' here refers to a local public space where we site the Pibrary
discussed in the previous chapter.

The Books of the Street are doing more than documenting existing
networks and businesses and people. They are \emph{creating} social
networks of actual humans in a physical space which do not exist in
today's globalized world of cars, planes, and long distance
communications. They represent a cultural shift to extreme localization
of communities localized to just a couple of miles across, which still
maintain the flow of global information across the entire human race.

Creating connections between people which did not exist before can
enable commerce. Enabling commerce creates cash flow in exactly the same
way it does in centralized commercial social media. This cash flow
generated by the network creates a strong incentive for network
participants to materially support network creators. Supporting us, the
network creators, allows us to spread the network, and if that spread
generates more value wherever it goes, that becomes self-sustaining in
growth. Initially, this network is simply a social media platform which
provides a totally free, non-capitalist (no money, no property) resource
to those in the existing capitalist economy. If we can scale with
positive cash flow in each local node, this creates a much more
efficient scaling mechanism than existing venture capital backed
technology startup companies, which generally scale at a loss in order
to gain market dominance. Without the billions of dollars of venture
capital money required to scale, we can move faster and be more
adaptable than those networks, enabling us to ultimately take away all
their market share bit by bit from a bottom-up approach.

Consider any ``technology'' company today which makes its money on
creating links between people. From ride share to dating to advertising,
all these companies are simply connectors. They connect people with
other people and then demand rent from us for doing so. A free network
driven locally from the bottom up with community owned hardware and no
intellectual property can easily defeat this network one street corner
at a time. We can take Silicon Valley down to zero if we can get the
right growth model of our network, and it is in the best interest of all
people that this happen as soon as possible, since the predatory model
of Silicon Valley is destroying us all. We are asking people with great
urgency to contribute to this campaign.

A final note on what books we choose to replicate on any given Pibrary.
In a traditional library, more is always considered to be better. The
more resources a library has the more books they buy, and it is assumed
that readers use search and browsing to find what they need with no core
specific purpose. This is not the case of the Pibrary. The Pibrary has a
purpose, our purpose is to create self-replicating media which can
transmit self-replicating technology made from trash which can provide
for all human needs for free. We therefore are very specifically
\emph{not} trying to just add more and more books. We want the selection
of books to be very aggressively curated by the caretakers of the
Network to specifically carry out whatever the next task is in any given
community to bring all of humanity closer to full Trash Magic. In some
cases, this might mean an individual person carries an individual
Raspberry Pi with just one or two books, specifically fro the exactly
actions of mutual aid and direct action in which we are engaged.
