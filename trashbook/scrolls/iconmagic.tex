
We live in an age when our basic understanding of reality is undergoing
radical transformation. We are also facing critical challenges like
climate change which can only be addressed by even more radical shifts
in our ways of thinking. This means we need to be able to think about
``things'' in the absolute most general possible sense and to work with
sets of abstract things in order to create radically new philosophies
which will allow us to undergo the needed changes.

In this section we introduce a general symbolic language for expressing
relationships between any collection of ``things'' in the most abstract
sense, using physical media which is part of our system of
self-replication. We use this system to design systems of
self-replicating sets. So we are creating a system of self-replicating
media used to describe self-replicating collections of things. This
forms a replacement for money. In a monetary economy, the most
fundamental things we exchange are always numbers. Money, stocks,
barrels of oil, ounces of gold, and so on are all exchanged in numbers
and represent a certain quantity of ``using up'' a finite resource. We
can in some sense think of this as ``number magic'', since the process
constantly replicates the desire to acquire more numbers, to
``replicate'' a number but only at the expense of someone else. The
media which carries all these numbers is generally either private files
in private databases or something like the block chain which publicly
tallies numbers. But in all cases the numbers are not allowed to be
copied. One cannot simply take a column of numbers with 10 units and
copy it over and over and give everyone 10 of that thing.

In contrast to this anti-copying property of money, in symbol magic, we
are using symbols to denote the sets of things we want to replicate, to
replicate the desire of replication. So to do this we need only the
absolute minimum complexity required to communicate this desire, and
then cause someone we are talking to to put in the effort to go copy a
thing. This thing might be an action or a physical object. It might be a
huge project which organizes thousands of people to work together or a
simple act carried out by one person. But always our focus is the
replication of the desire to carry out some physical action. At some
level this is a form of advertising, of brand manipulation, and these
icons can indeed be corporate logos or brands.

This represents an economic system. Again, in a monetary economy our
media is just the list of numbers and people. In this system our media
is self-replicating physical media which represents things we wish to
replicate. In a monetary economy the fundamental transaction is an
exchange of numbers. You get some of one number and I get some of
another number, each of us gets a debit of one thing(like dollars) and
each gets a credit of another thing(like gallons of gasoline).

In a symbolic replication economy I show you a symbolic representation
of a thing we wish to replicate and do whatever I need to do to
replicate the replication. That is to say, to communicate to you what
you need to go replicate it somewhere else. This might be as simple as a
domain or hashtag or social media handle which points you to
instructions to copy. Given the Web's ability to create deep knowledge
if you know how to find it, this symbolic communication only needs to
point to things and replicate intent. The deep knowledge can all be in
online documents(all of which also self-replicate).

We present an example of a symbolic replication economy here we call
``Icon Magic''. In Icon Magic, we create self-replicating Geoemtron
glyphs in the web browser which both create symbols in the browser and
also create programs which can run on robots made out of trash to print
in physical media. We use this with a simple printer robot made of three
DVD drives to print in clay using a nail poked repeatedly in an
arrangement of pixels. The Icon design software in the browser has a
simple system of tracing over images found via web image searches, so
that no real artistic skill is needed. Also, the only command needed to
create a glyph are up, down, left, right, and the same movements
combined with drawing one pixel, so a full understanding of Geometron is
also not needed, and this technology can be used by people with no
common language or technical skills, just by point and click or simple
keystrokes.

We print in Sculpey polymer clay, which can bake in a regular home oven.
Once a print is baked, a stamp can be made in it's mirror image with
another blob of Sculpey. After this is baked it can be used to stamp out
a copy of the original print. This process allows one print to make many
stamps and each stamp to make many copies, so one print can replicate
out to hundreds or even thousands of copies. This is what we mean by
self-replicating media. Anyone anywhere in the world can create a
Geometron glyph on any server, which they can then replicate an infinite
number of times to every other server in the world. Each one of these
can then be printed on a robot which is itself made from trash and
documented for replication on all our servers. Each print can then
replicate out to thousands of final tokens. These tokens are then
painted and sanded, so the paint stays in the dimples, and complex
colored symbols can be created this way. Note how this is the opposite
of money! Money takes its value from its inhibition of replication. If
everyone in the world can copy as many 20 dollar bills as they want, it
becomes worthless. But in our system, value comes from replication, and
the more people copy a piece of media the more it is worth, because it
pumps energy into the rest of our replication system from which we
derive all value.

Once these self-replicating icon tokens have been created, we can use
them for many things. They can represent objects, people, ideas, places,
game pieces, symbols, actions, brands, or collections of any type of
``thing'' in the most abstract sense. We use them to communicate
descriptions of relations between groups of things to talk about
replication of things. To do this, we need to arrange them on some
surface, and to do that we turn again to Action Geometry as described in
the last chapter. We can create a generalized ``board'' using sharpie on
cardboard with our basic set of geometric shapes. This can be thought of
as a generalized table top game made from self-replicating media.
Self-replicating shapes are used to make the same pattern again and
again on cardboard trash, on which the same arrangement of tokens can be
displayed and manipulated to communication replication to other people.
This physical media is used in a public space covered by the Pibrary. In
other words a space with physical media pointing to free domains, along
with a free wireless network, free off grid power and free
computers/servers of the Pibrary hosting free books which replicate the
whole system here.

This connection to the Pibrary is what makes this a functioning economy.
If I'm in a high traffic public space with physical media which draws
people in and digital media which mirrors all documents to the public
Internet, I can communicate all the details of how to replicate complex
technology. If I have a space to use our universal philosophical
language of Icon Magic on Action Geometry boards to communicate the
desire to replicate sets, and the cardboard signs to point people to the
digital media, that is a full system of replication. If that system is
all on cheap off-the-shelf hardware and self-replicating software, the
whole system can replicate. We are building a system where
\emph{everything} replicates. In such a system property and money don't
really make any sense, because they only function when replication is
inhibited.

If such a system produces enough value for people to live on completely,
it will create an incentive to transfer more and more material objects
into this system. That acts as a force of Nature, naturally transforming
property into non-property, and money-based transactions into
replication-based transactions.

It is worth illustrating this concept with the mechanics of replication
transactions as compared directly with money transactions. Suppose I sit
in a public space with a sign. That's the same in both systems. You see
the sign in a public space and see me sitting there and come over and
sit down and talk in both systems. Now in the money system, we discuss
products, agree on a price and exchange money for products. In the
replication system, we discuss things which are being replicated, which
might be products, actions, ideas or anything. This discussion is
mediated by the ``game boards'' which are used to place icon tokens as
if we are playing a game. We can both move them around and talk about
sets of things and actions to replicate them. Media can be exchanged in
the form of cardboard or paper with addresses of media resources or
direct exchange and use of links on mobile devices. Tokens can be
replicated with clay or with other malleable physical media(like poured
resin or silicone) and boards with more cardboard trash and our basic
geometry tools, and the set carried by both parties after the
transaction to replicate again and again. So a network of such stations
can replicate sets with exponential growth(like a virus) across all of
humanity with almost no physical global supply chains. All of this is
based on cardboard signs in public spaces, of which there is already a
network along all the roads of the world, we are merely adding layers of
free media to this existing network.

The various types of clay pieces in the system are all stored in a set
of three sewn cloth bags. We think of these bags as having various
symbolic meanings, of Earth for the prints, Fire for the stamps, and
Water for the final icon tokens we use for communication. So to make our
system fully replicating when we are out in public we need all three
bags. The prints are used to make stamps, and the stamps are used to
make the final tablet sets, which are carried in the ``water bags''.
These bags are sewn from black cotton flannel out of rectangles 8.5x13
inches in size, with an 18 inch black nylon parachute cord sewn into the
top as a draw string. The ``Earth'' bag has a three inch green felt
square sewn onto it. Fire has a three inch red equilateral triangle with
the point up, and water has a three inch blue equilateral triangle with
the point down. We also need sanding blocks or sand paper, and paint
pens, and access to a conventional oven, as well as access to a printer
robot made from trash and an Arduino, also carried around in a bag. This
whole set of physical things can be carried into our public spaces, and
used to replicate itself completely with passerby.

The stamps in the Fire bag can also be used to stamp into plastic which
is melted over an open candle. This plastic can then be colored in with
paint pen and sanded flat like with clay, to make an infinite number of
prints with one stamp onto plastic trash. When we combine plastic trash
printed with arbitrary symbols with cardboard trash printed with
arbitrary geometry from Action Geometry, we have a created a media for
expressing any concept of human thought using trash. And this media is
physically integrated into a space. Also, all the media points to online
resources which link all the parts of the physical space. This is a
philosophical language for mixed reality social media.

All this might sound somewhat abstract, and must be illustrated with
some examples. The first example is replicating the system itself. This
means we just want to have icon tokens for each thing discussed in this
book. For example, the Raspberry Pi, solar panels, batteries, flags, and
so on. Each thing in this work which has enough of an independent
identity in discussion that it's worth talking about gets a symbol. We
then make generic boards which are just attractive geometric patterns
which put some kind of structure on the cardboard and give it that
distinctive geometric pattern which is easy to recognize.

Another example is just game pieces, which are actual physical products
with value which are replicated along with everything else. We can make
self-replicating chess sets which are carried in a bag and used on
cardboard chess sets with sharpie based squares. Each chess board can
act as an Action Geometry shape to replicate and make another chess
board on more cardboard. As with all sets, we have three bags, and can
use the prints to replicate stamps and the stamps to replicate pieces.
So one printed set on a robot can create thousands of fully functional
chess sets, and whole networks of people replicating the sets, playing
chess with them, and replicating them again.

We also use the tokens as game pieces in the mixed reality environment,
placing them in various locations in a public space. This mixed reality
can involved cardboard with geometry and web addresses, hash tags or
contact info and game tokens, all left in public spaces linked to by
public facing web pages. This represents a complex network of media
which is always a hybrid between physical and digital and all outside of
the property system, left in public without any personal possession. All
of it is self-replicating, as pages all replicate from server to server,
tokens replicate with clay and cardboard replicates with Action
Geometry.

We may think of this system as a philosophical language, a universal
system for representing structures of human thought. This is something
that various philosophers have worked on in the past, but our goals are
different. When philosophers like Gottfried Wilhelm Leibniz worked on
this problem the goal of all that work was still to express ``truth''.
The goal of both philosophy and science in that time was to create as
many statements as possible which were as true as possible. This is not
our goal. Our goal is to create the collection of information which
taken together allows us to create a fully self-replicating system of
technology from only trash, the sun, water, and the living Earth for all
people to live a good life for free everywhere. We will create the
structure of our linguistic and philosophical tools around this.

We have created a universal symbolic language as a tool for creating a
new mode of human existence based on replication instead of mining. This
is not exactly ``technology''. This system is philosophy, it represents
an approach to interacting with each other and with things, not a
specific technology. It could easily be replicated using methods from
thousands of years ago with clay and sticks. Indeed, we can probably
think of early human stone tools as examples of self-replicating media
in this way. When the first people figured out stone tools, in order for
that to replicate enough to have a global impact on humanity they have
to travel in a replication economy. One person chipping stones in one
creek bed with one special type of stone doesn't scale. A culture of
stone-chipping replication does scale. And each spear with a stone point
is media which advertises its own replication. Its product in the form
of animals to eat naturally replicated the desire to replicate the
thing.

This type of media and economy was consumed by the mine system
everywhere in the world as the mine-users created machines of war, used
them to get more land for more mines and competed to keep doing that
until the whole world is one giant mine feeding one giant war machine.

This book must itself be part of a self-replicating set using Icon Magic
and Action Geometry. Every single thing described here must be
replicated this way, including this book itself, which documents
replicating all the things. Everything is recursive in that it points
back to itself by replication. Replication can be thought of as a type
of ritual. Rituals are sequences of actions with meaning. We will
integrate the rituals of replication into existing cultural frameworks
by mixing whatever is already replicating in a given community(religion,
culture, customs, commerce) with the replication of Trash Magic and
Geometron. We do this by representing all the things already in
existence in any given community in Icon Magic, and creating ways to
represent those things using tokens on boards. This can include
religious ceremonies, divination and performance art, business deals,
relationship and network building, art projects, games, buying and
selling, or really anything anyone might possibly want to do.

Language is how the mind parses reality. Therefore the most fundamental
thing which determines how we connect our minds to reality is the
structure of language. Building a symbolic language the sole purpose of
which is to create a replication economy on trash, the sun, water and
the living Earth represents a shift at this deep level where our minds
connect to reality. Doing this with cardboard and sharpie puts this
linguistic tool in the hands of the people we need to help the most,
those who are the most marginalized. This is a language in which mutual
aid and direct action are hard coded into the structure by making
everything free, focusing on those who have the greatest need, and
directly and freely replicating whole system again and again. If this
language is able to replicate along with this network, we can consume
the old economic system of money, mining and property as fungus consumes
a log, turning all these things into new things which can stay put and
cycle materials freely forever.
