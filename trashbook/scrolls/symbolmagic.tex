
Symbol Magic, like all the ``magics'' of this work, refers to
self-replicating symbols, which people freely copy using the media
network we are building. All our media is designed to be created,
edited, and replicated entirely in a web browser. The media we use to
control \emph{all} machines in full Trash Magic is to be built this way.
Be it a cutting tool, a laser, an agricultural robot, pixels on a
screen, or any other movement of any machine used to impart form onto
matter, we aim to create all the programs to control this in the browser
using symbols also made in this system.

As with all our systems, this begins with the ability of the system to
create itself. This means we need to be able to make the web graphics
used for buttons and links in our web-based media. If you're using this
system to read this book, you have already seen these. Using simple
square symbols to represent things they link to or actions they cause to
happen is a fundamental element of what makes the web work.

We also must now take a moment to define ``symbol'' in the broad sense
used in this work. We define ``symbol'' to be \emph{any} geometric
construction which has meaning to people. This means a building is a
symbol. A pie is a symbol. A culvert or dam is a symbol. A
microelectronic circuit is a symbol. All of these things have meaning to
people and are constructed with geometry. If the materials used to make
a symbol are available locally everywhere in the world and the symbol
can be replicated from one web browser to the next, those things can all
be replicated. If the system of symbols, which we can think of as a
language, also impart the \emph{desire} to replicate a thing, that is
where the magic happens, where the things can spread freely from one
local community of people to the next across the globe.

The language of Geometron in the web browser is more completely
documented in the First Book of Geometron. Here we just discuss how it
works and what it can do and what we plan to do with it as the system
grows.

Symbols used to represent geometric actions are generally in a square.
The symbol itself is a sequence of geometric actions. So for instance
the action ``draw a circle'' is denoted by a square with a circle and
dot inside it. In order to draw that symbol, we must instruct a browser
to draw pixels representing this geometry. We do that by breaking down
the construction into actions which can be described in a purely
geometric way, without direct reference to numbers. For the symbol for
circle, for example, the construction is: move to the right, draw a
square, shrink the unit by two, move left and up, draw circle and dot,
move down and right, then increase the unit by two. This unit is of
course some number of pixels in practice, but the construction uses
language independent of the actual value of that unit. This makes our
language independent of specific numbers. It is a sequence of discrete
geometric transformations, which can be carried out with any physical
medium at any scale in any coordinates. This abstraction is incredibly
powerful because it means that we can very quickly and easily create a
new implementation in any language or format, and transfer a symbol over
without changing the code. This is how a symbol designed in a web
browser can be converted to a physical thing made in a laser cutter, 3d
printer, clay printing robot, spray paint robot or microscopic
lithography tool.

The basis for all the discrete geometric actions are scales and
symmetries. We do not refer to angles in degrees or distance in pixels,
but in rotations and resize operations based on the natural structure of
the world. This means we start with fourfold symmetry, fivefold symmetry
and sixfold symmetry. Rotations are combined with angle manipulations
like doubling or bisecting an angle, or tripling, or dividing by three.
This approach uses numbers in any given implementation, and can be used
to represent numbers but again is not actually based in numbers. The
symbols we use denote actions based on symmetries, rather than any
specific reference to angles in numbers. The scaling is based on the
symmetries. For example, everything based on the pentagon and pentagram
have natural scaling based on the Golden Ratio, which is the ratio of
the side of a pentagon to the longer distance between points across the
pentagon. The same relationship exists for the hexagon and square root
of three scaling, and the 45 degree right triangle and the square root
of two. So rather than using numbers to re-scale units, we use these
universal scale factors based on symmetry, like ``multiply unit by the
square root of three'' or ``divide by the Golden Ratio''.

We use this system of drawing and moving a drawing tool around to create
\emph{everything}. This can be used as a replacement for both art and
engineering software, creating plans that people can read to build
things with their hands. In its most basic form, it is used to
manipulate pixels in a web browser using the ``canvas'' element which is
part of the basic standard available in all web browsers via HTML and
JavaScript. We can interact with Geometron with a touch screen, a
keyboard, or other hardware interfaces we build based on Arduino. In the
touch screen, we work in the web browser as always, and use buttons
created out of canvas elements which have events tied to them which
control the ``Geometron virtual machine'', or GVM, which is part of the
code in our system. We can also use a keyboard to control the actions of
the GVM by putting the cursor in a text input and writing keystroke
events which call our code and do things to the GVM and ultimately
symbols in canvas elements. This shows how everything in our system
points back to itself(everything is recursive). A canvas displays a
sequence of symbols, each of which is drawn with a piece of software
which replicates with scripts which also run in the browser. This
sequence is edited by hitting keys on a keyboard painted with symbols or
canvas elements which use the same software to display the symbol of the
geometric action in our program. Any sequence of Geometron actions can
be called a ``glyph''.

A Geometron glyph is a magic symbol in the sense that it replicates
itself with human control. We edit the symbols which control all our
geometry using symbols made of symbols and so on. All this can happen in
a web browser using applications which replicate via the cybermagic
system discussed earlier in this work. All the information required to
make a GVM for a canvas and edit glyphs is contained in a JavaScript
library called geometron.js, which we replicate and edit as part of our
system.

The software which we use for all the work with canvas elements can also
export to the vector graphics format SVG(for Scalable Vector Graphics),
as well as the bitmap format PNG(Portable Network Graphics). The icons
used as links and buttons in our system are in the SVG format and are
all stored in a directory which gets listed and copied using dna.txt and
replicator.php in the cybermagic system. These can also be used as
technical illustrations and art in books in our system. Collections of
these files are in a feed which we generate using web based applications
in cybermagic, and these collections are themselves self-replicating
sets. Replicator scripts can be run from in a browser on any server
which will replicate a set of SVG files from any other server on the
entire Internet. This set can then all be edited live in the browser on
the new server, and then replicated out to yet another server and so on.
This is the power of Geometron Symbol Magic: to have symbols be edited
live in any browser on the planet from any server, and then replicated
from server to server again and again, evolving freely. Every icon in
our system is like this, and can be edited and changed as the system
evolves.

These SVG and PNG files are also the basis of physical fabrication. They
can be used to create physical objects using laser cutters. This can be
done on a wide variety of laser cutters, or they can be uploaded to a
print-on-demand laser cutter service like Ponoko.com or a public library
maker space. Laser cutters can cut and etch a huge range of materials,
including a lot of waste materials. Cardboard, plastic, sheet metal and
wood can all be cut into shapes in a laser cutter. This means we already
have a system here by which self-replicating files entirely managed from
inside a web browser can create physical objects out of trash! This is
trash magic! The SVG file format can also be imported into other
software used to control machines which make physical things, like
electron beam lithography for making very small electronic circuits or
embroidery machines for making textile patterns.

This system of discrete geometric movements and constructions can also
be used to construct three dimensional files from a web browser. We use
several three dimensional web-based file formats to do this. The same
canvas element we use for all our two dimensional graphics editing in
the browser also has a three dimensional mode which we can control with
Geometron for live editing. The format formerly known as VRML for
Virtual Reality Markup Language, is now called x3d, and a Geometron
glyph created in the canvas can be exported to this format, which can
then be imported into numerous types of 3d software like virtual
reality, augmented reality and games. There are also web libraries which
export to the STL format used by 3d printers. With these standard file
formats we can get from the web browser to numerous engineering software
systems which allow things designed in our system in the browser to turn
into physical objects using numerous machines.

When we build our own fabrication machines from trash, we control them
all using the open source hardware platform Arduino(for now). In order
to be able to do all programming from the browser, our system includes
the ability to generate Arduino code from the symbol glyphs we create in
the canvas element which is printed in a text area by our software and
which can be copy/pasted into the Arduino software to load onto the
board without ever having to interact directly with the Arduino code.
This makes the machines we build much more accessible to far more people
than is ever possible in existing systems of machine control. Rather
than learning to control a machine with either low level code made of
numbers and broken English or some high level system based on
specialized applications for some specific hardware, we allow anyone
anywhere on the Internet to create, edit and share the glyphs made up of
symbols which determine what a machine will do.

As an example we imagine a machine in which a winch is on a rail moving
side to side across the top of the wall of a high rise apartment
building. We build a language of symbols which denote ``move left one
unit'', ``move right one unit'', ``double unit'', ``halve unit'', and so
on, all of which display in a canvas element in the browser. These
symbols are also painted on keys on a physical keyboard used to input
keystrokes into the browser. A non expert can write a sequence of
symbols which create a sequence of physical actions in an intuitive way.
Then, if the robot is rebuilt with a totally different motor and control
technology, whoever builds the new system only needs to find a way to
implement ``move right one unit'' and so on into the new system and the
glyphs written in the old system will work on the new one with no
modification.

As our system develops we will replace more and more parts with
Geometron, until all of our machines are based on self-replicating
symbols, whether they are for communication or fabrication or machines
which carry out some other task. Ultimately we will build a system by
which symbols are the medium of replication we as humans use to
replicate all technology from trash and living material forever. The
path into the future of this development will be developed in a later
chapter in this work.
