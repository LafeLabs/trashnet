
Trash Magic is a mode of existence in which we can replicate everything
we need to live a good life locally using the waste streams of the
existing consumer system. We are using the word ``magic'' as in the rest
of this work to refer to the replication of the desire to replicate
things made from trash.

Full Trash Magic is the ultimate objective of all this work. Under full
trash magic, all people, everywhere in the world, can get food, shelter,
medicine, media, sanitation, water, heating and cooling, and the
machines to produce all these for free locally. We will abolish all
mining, all oil and gas extraction and all global physical supply
chains.

In order to achieve this objective we begin small, with something that
immediately provides value, and scale up based on replication of the
thing which provides value. If we can do even the simplest thing which
just barely provides a small amount of value from trash but
\emph{replicates} and \emph{evolves} with intent, we can then simply
guide that evolution and growth to navigate to the complete system as we
engage more and more people with more and more specialized skills and
resources.

To start all this, we turn to the industrial revolution as a guide. Much
of what powered the industrial revolution was using new energy sources
in the form of coal and steam to build machines which build other
machines. Also, textiles have always played a central role in technology
replication, as their products become central to people's culture, which
replicates and brings the textile production machines along with them.

In analogy to all this, we want to build the smallest possible factory,
which we call a Trash Factory, which mimics this pattern but without
mining. We want to build machines that can build machines, or a machine
shop, powered by the local forces of the Sun, wind, and flowing water. A
machine shop is a collection of tools which can work metal into the
forms needed to make more machines. Machine shops are how metal machines
traditionally replicate themselves. We need to be able to melt metal
waste into metal ingots then process that into bars, sheets, rods, wires
and blocks. Then we need milling machines, lathes and drill presses to
machine them into desired shapes. We need the tools of sheet metal work
like the brake and bender. We need an arc welder, torches and some other
basic tools for soldering, welding and brazing. All this must be made
from trash.

Building a machine shop can be based to a large extent on junk cars.
Cars have plenty of steel, plenty of parts to salvage without any
melting or casting, and electrical tools which can be used for motors
and so on. As much as possible we will use things as we find them
without reprocessing. If we can, we'll just get donated old stuff that
is broken and fix it. The machine shop maintained by people good at
fixing broken stuff is as old as the industrial revolution, we just aim
to build this into the rest of our self-replicating media system.

The machine shop also needs to have tools for working plastics, with
molding on metal molds created using the metal shop, and plastic welding
and rework tools. An electrical shop is needed for electric motors and
generators.

A fully functioning machine shop which is optimized to build more
machines from junk cars can be a self-replicating and self-sustaining
factory just by selling machines. We can sell drill presses, milling
machines and the like for money which can support the people who build
and maintain the system.

In addition to the machines which replicate themselves, we will build
all the tools for creating trash-based clothes on site. We will build or
fix broken sewing machines, and use them to create fashionable and
functional original clothing of all kids for all people for free to
those in the most need. If our story replicates as we hope it does, and
people believe in our mission, we should be able to support all the work
to build the system, to operate it, and to deliver the free clothes to
those in need by selling high end fashion to those who can afford it.
All clothes are made on site with waste clothing donated from those in
physical proximity to the Trash Factory.

All the motive power for Trash Factories is provided by one of three
main sources: heat engines, water drive, and wind. An essential
technology which must be integrated into the first generation of Trash
Factory is the trash-built Stirling Engine. This is a very simple heat
engine developed in the 1800s and used widely ever since which can turn
heat into mechanical motion by compressing and decompressing a gas in a
sealed chamber with a piston. These engines have been overshadowed by
the internal combustion engine or the giant steam turbines used in large
scale commercial power plants, but they work well and are well
understood and simple. The primary means of driving a heat engine in
Trash Magic(without setting things on fire) is using the energy of the
sun focused via mirrors onto a heat absorber. Large arrays of mirrors
can be built from trash which track the sun and maintain the focus of
the sun over a large area onto the absorber. The other robotics
technology that is part of Geometron can be used to steer the mirrors as
the angle of the sun changes. Stirling engines can also be run
backwards, creating a heat pump when the shaft is turned. This means
they can be used to cool things, being the basis of solar-powered air
conditioning and solar-powered refrigeration. Solar powered air
conditioners sound almost too good to be true, but this has been
demonstrated well over 100 years ago, it is just not used today for
economic and social reasons. The heat engines are also a very good
commercial product which can be sold(as an off-grid power source) for
money to support the rest of Factory operations.

Water and wind are both pretty traditional: we simply build rotors for
both from trash and source the drive trains and generators from trash.
Water can be waves, tide or streams/rivers/creeks. In all cases, we
envision a factory which has between 1 and 5 people operating it at a
time using between 1 kW and 100 kW.

The absolute maximum available solar power in direct sunlight on a clear
day is about 1 kW per square meter so at 100\% efficiency(which will
never happen) this is up to about a 100 square meters. If we imagine
getting a pessimistic 10\% efficiency, that's up to 1000 square meters,
which is about 30 meters on a side or about 100 feet on a side
square(about a quarter acre or 0.1 hectares).

A reasonable site for a Trash Factory will be about 1 acre, or about
4000 square meters or 0.4 hectares. This will be enough space for a
machine shop, the power station, and the various staging areas we need
for incoming waste streams and outgoing product streams. When possible
we site near flowing natural water and use extra power to both pump
water uphill and to clean it up for drinking. Water can then be both
used to drink and used to get energy back out as it flows downhill from
a water tower or hill top reservoir. Our goal is to be a very
scaled-down version of the River Rouge factory from the Ford Motor
Company from the early 20th century, where a constant flow of raw
trash(instead of raw material from a mine) comes in one side and a flow
of finished manufactured goods flows out the other side.

The Trash Factory can be sited based on convenience to resources, cheap
land zoned for heavy industrial activity, and easy access. It does not
have to be an ideal retail location. The retail side of the Trash
Factory is free stores and existing shops. We can make things to
directly provide for free for those who want, providing warmth and
protection with fashionable and well-fit clothes sourced from local
trash while also sourcing products for local stores shelves we sell for
money to support the Factory. This also applies to all the machines
produced in the machine shop: we can sell welders at a welding shop,
heat pumps through an HVAC(heating, ventilation, and air conditioning)
distributor, drill presses and machine tools to auto shops, etc. Also,
providing a mix of free and commercial products to our local community
creates the human relationships we need to establish to keep our supply
chain flowing of trash we get for free from existing waste streams.

Again, the Trash Factory aims to always produce more value than it
consumes, both bringing in enough cash to support the people operating
it and the land and also providing material support for whoever is the
most wanting in the local community. Every kilogram of mass we convert
from trash to products locally takes that kilogram of mass out of both
the landfill waste stream and the mine stream of consumer society. If we
can make this replicate and evolve, we can keep removing more and more
energy from that system over time, and pumping more and more energy into
our system. As long as replication of this system takes less energy than
replicating the existing systems we will naturally consume the old
system for reasons of simple thermodynamics.

But where does ``Trash Magic'' fit in with all this? Trash Magic refers
to the transmission of this system of trash-built and trash-sourced
factory using the self-replicating media platform described in this
work. Every machine, article of clothing, every clever hack and
structure of business or organization will be documented in a library of
books(including this one) which are kept on free media and network
infrastructure we build into all of our systems. In the beginning this
will be the Raspberry Pi(a very cheap and open source computer) based
system which starts building our network, along with off the shelf
commercial wireless network infrastructure. As we develop our system it
will evolve into the fully trash-built media described later in this
work.

Full Trash Magic involves taking the Trash Factory system described here
and scaling it up to all things we need. As we grow we always direct all
excess value created by the system into helping the most needy in the
immediate physical community around the Factory. As this pulls more
energy into the system, we will be able to get access to more land and
resources outside the property system. Directing resources to those with
the most needs first will abolish poverty in very localized areas.
Abolition of local poverty will enable more space outside of the
property system to flourish, on which we can create products which are
all outside the property system.

This will ultimately include the whole set. We need to build fresh water
generators from dirty water, and build a toilet infrastructure which
turns human waste into compost which is used to grow things locally
including organic fiber crops for toilet paper(which can also be a
product of the Factory). This waste disposal and composting system will
be integrated into a system of local synthetic biology, where we use
modern biotechnology techniques to control microorganisms and fungi to
make all the medicine we need on site. Again, building bioreactors which
can make all medicine is nothing new, we just need enough human energy
in the system that we can attract the talent in the form of experts who
already know how to build such systems. Our aim initially is not to
invent anything, but to create the social connections which allow people
with expertise to connect with real local community needs and then to
scale that through self-replicating media. If we can make clean water,
machines, clothes, medicine, food, and media on site, we have a system
which can sustain human life without mining or oil as is our goal. You
can think of our whole social media system as like a ride share app but
for finding the people with whom we build a sustainable civilization.

Again, we must reiterate that this is not some futuristic hypothetical
technology. Creating free web pages on free computers which tell you how
to make things from trash is simple. Making things from trash is known.
The waste is plentiful. 300 years of industrial production brings the
whole periodic table of elements right to your doorstep. The needs of
the most impoverished and marginalized people in any given local
community are known. The mass peer-to-peer media network of the Internet
exists which can spread all of this. All we are saying with this work is
that these dots can be connected. The only thing missing is the
\emph{will} to connect these dots. And of course while we might already
have the will to do it, what we need to make it scale is the ability to
create specific detailed plans and replicate the desire to carry those
out. The media platform documented in the rest of this work will allow
us to do this. The revenue we will generate by simply building the free
social network of free books will provide the startup capital(not
financial capital, but resources like land and human attention) to build
our first Factories.

Full Trash Magic can exist on just a few acres with just a few dozen
people. We can achieve this in our lifetimes if we focus on our
objective and work together!
